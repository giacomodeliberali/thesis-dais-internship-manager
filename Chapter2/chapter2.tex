\chapter{Architettura}

Questo capitolo illustra il funzionamento e la struttura delle architetture software, sia per quanto riguarda il \gls{frontend} che il \gls{backend}. Dal momento che entrambe le applicazioni \angular~e \nodejs~operano sugli stessi oggetti, le loro definizioni risiedono in un pacchetto \gls{npm} condiviso che esporta tutte le entità necessarie, in modo da riutilizzare il codice e renderlo facilmente manutenibile. Le principali entità esportate sono:
\begin{itemize}
	\item \textit{User}: rappresenta un utente del sistema con uno specifico ruolo
	\item \textit{Company}: rappresenta un'azienda
	\item \textit{Internship}: rappresenta un'offerta di tirocinio di un'azienda
	\item \textit{InternshipProposal}: rappresenta una proposta di candidatura di uno studente per un tirocinio
\end{itemize}

\section{Architettura lato server}

Il \gls{backend} di \projectName è un'applicazione \nodejs~che si appoggia sul \gls{framework} \expressjs~per l'esposizione di \acrshort{rest} \acrshort{api} che interagiscono con \mongodb~attraverso l'\acrshort{orm} \mongoosejs. \\

\noindent
L'infrastruttura server è divisa in diversi livelli con responsabilità diverse che interagiscono fra loro che verranno discusse nei paragrafi seguenti.

\subsection{Schemas}
Pur utilizzando \mongodb~che è uno schema-less \acrshort{dbms} ho preferito appoggiarmi su un sistema che mi permettesse di validare i dati e la loro struttura. Per fare ciò ho utilizzato \mongoosejs, un \gls{orm} che mi permette di definire la struttura dei documenti nelle collezioni del database, mi fornisce metodi di validazione e manipolazione basati su oggetti.

Gli \textit{schemas} sono la definizione dei documenti delle collezioni del database come previsti da \mongoosejs. Essi contengono la definizione del documento, dei suoi campi e del loro tipo.
% File
\begin{figure}[!h] 
	\centering    
	\lstinputlisting{Chapter2/schema.ts}
	\caption[Esempio di \textit{schema} dell'applicazione \gls{backend}]{Esempio di \textit{schema} dell'applicazione \gls{backend}}
	\label{fig:server-schema}
\end{figure}
Possiamo notare che in figura \ref{fig:server-schema}, all'interno della definizione della struttura del documento, vi sono tre proprietà --- \textit{attendances, internship e status}. Ognuno dei campi ha tipo diverso:
\begin{itemize}
	\item \textit{attendances} è un array di oggetti con una properietà \textit{date} di tipo \textit{Date} obbligatoria
	\item \textit{internship} è una referenza di un altro schema (Internship), che verrà popolato automaticamente in fase di lettura (effettua un \gls{sqljoin} in automatico con il plugin \acrshort{npm} `mongoose-autopopulate`)
	\item \textit{status} è semplice numero
\end{itemize}
Nel caso l'applicazione cerchi di salvare un oggetto che non rispetti i vincoli imposti dallo \textit{schema} viene sollevata un'eccezione che impedisce di rendere inconsistente il database.

\subsection{Repositories}
I \textit{repositories} sono classi legate ad uno specifico oggetto che esportano operazioni su di esso. Interrogano uno o più \textit{schemas} per leggere, scrivere o aggregare dati, e contengono solamente la logica di accesso ai dati, senza nessuna logica di business (ad esempio il controllo dei permessi). Tutti i \textit{repositories} dervano dal \textit{BaseRepository} che esporta le operazioni di base --- \gls{crud} --- oltre che un metodo per eseguire query personalizzate.
% File
\begin{figure}[!h] 
	\centering    
	\lstinputlisting{Chapter2/repository.ts}
	\caption[Esempio di \textit{Repository}]{Esempio di \textit{repository} dell'applicazione \gls{backend}}
	\label{fig:server-repository}
\end{figure}
Ogni \textit{repository} può esporre ulteriori metodi personalizzati ed eventualmente accedere e ad altri \textit{repositories} iniettati dal sistema di \acrfull{di}. Il sistema di \acrshort{di} adottato dal sistema si basa sul pacchetto \acrshort{npm} `inversisy`, che fornisce anche un modo di aggirare le dipendenze circolari (ad esempio tra due \textit{repositories}) tramite un meccanismo di \textit{lazy-inject}.


\subsection{Controllers}
\label{server:controllers}
I \textit{controllers} espongono su architettura \acrshort{rest} \acrshort{api} un metodo di \expressjs~ che internamente utilizza le operazioni dei \textit{repositories}. Anche i \textit{controllers} sono legati ad un unica entità e dervano da un \textit{BaseController} che di default espone i metodi per le operazioni di \acrshort{crud} dell'entità. Tutti i metodi esposti ritornano il risultato wrappato all'interno di un oggetto, \textit{ApiResponseDto}, che contiene anche informazioni aggiuntive, come lo stato \acrshort{http} ed eventuali errori.
%
I \textit{controllers} sono responsabili di gestire l'autenticazione dell'utente e le eventuiali eccezioni sollevate dai \textit{repositories}. Ogni \textit{controller} può esporre metodi che richiedono ruoli diversi, quindi ogni metodo definisce tramite uno \textit{\hyperref[server:scopes]{autentication middleware}} quale siano gli utenti abilitati ad eseguire quel metodo e in caso negativo ritorna un errore di autenticazione.

\begin{figure}[H] 
	\centering    
	\lstinputlisting{Chapter2/controller1.ts}
	\caption[Esempio di registrazione di un metodo \expressjs~in un \textit{Controller}]{Esempio di registrazione di un metodo \expressjs~in un \textit{Controller}}
	\label{fig:server-controller-1}
\end{figure}

\noindent
Come possiamo notare in figura \ref{fig:server-controller-1} il \textit{controller} sta registrando una route di \expressjs~il cui  secondo parametro è un array. Queso array contiene un insieme di \textit{middleware} che \expressjs~si occuperà di invocare prima di eseguire il codice all'interno della \gls{callback}. Il \textit{middleware}, in questo caso la funzione \textit{ownInternshipProposal}, verifica che l'utente che ha effettuato la chiamata sia effettivamente un soggetto (azienda, professore o studente) della proposta di tirocinio di cui vuole aggiornare lo stato. In caso negativo rifiuta la richiesta con un errore e la termina. L'autenticazione del sistema verrà discussa in dettaglio nel capitolo \textit{\ref{chap:api} - \nameref{chap:api}}.

\subsection{Bootstrap}

L'avvio dell'applicazione avviene nel file \textit{server.ts}, che si preoccupa di registrare tutti i \textit{repositories} nel sistema di \acrlong{di} e infine di rispolvere i \textit{controllers}.
\begin{figure}[H] 
	\centering    
	\lstinputlisting{Chapter2/server.ts}
	\caption[Estratto di \textit{Server.ts}, bootstrap del \gls{backend}]{Estratto di \textit{Server.ts}, bootstrap del \gls{backend}}
	\label{fig:server-bootstrap}
\end{figure}

\noindent
L'applicazione si preoccupa anche di connettersi al database \mongodb~e registrare nel container l'istanza dell'applicazione in modo sia accessibile dove necessario. Definisce inoltre un \textit{middleware} per catturare eventuali eccezioni non gestite dai \textit{controllers}.

\pagebreak
\section{Archietettura lato client}

Il \gls{frontend} di \projectName~è un'applicazione \angular~(v6) composta da diversi moduli (\nameref{chap:ngmodules}). Vi sono due moduli principali, uno che contiene le pagine che può visualizzare un utente non autenticato (\textit{NoAuthModule}) e uno che contiene le pagine visibili agli utenti autenticati (\textit{AuthModule}). Il modulo per gli utenti non autenticati viene caricato per primo, mentre il modulo che contiene le pagine protette viene caricato con la tecnica del \textit{lazy-loading} solamente una volta effettuato il login. Esiste poi un altro macro modulo, \textit{SharedModule}, che contiene dipendenze utilizzare in entrambi gli altri due moduli e che viene infatti importato in essi.

\subsection{Moduli e divisione delle responsabilità}
\label{client:modules}
\subsubsection{SharedModule}
\label{client:shared-module}
Contiene le dipendenze, i \textit{components}, le \textit{pipes} e le \textit{directives} utilizzate negli altri moduli. Viene importato sia in \textit{NoAuthModule} che in \textit{AuthModule}.

\subsubsection{NoAuthModule}
\label{client:no-auth-module}
Contiene la pagine che solo un utente non autenticato può raggiungere, ovvero \textit{login} di un utente e \textit{registrazione} di un'azienda. Una volta effettuato il login, il \textit{router} carica il modulo \textit{AuthModule} che contiene invece tutte le pagine protette e fintanto che l'utente non effettua il \textit{logout} non può più raggiungere le pagine in \textit{NoAuthModule}.


\subsubsection{AuthModule}
\label{client:auth-module}
Contiene la pagine che solo un utente autenticato può raggiungere, ovvero il cuore dell'applicazione. Data la sua dimensione questo modulo contiene degli altri sotto moduli:
\begin{itemize}
	\item \textit{UserModule}: contiene la parte di gestione dell'account di un utente, con la possibilità di modificarlo
	\item \textit{Internship}: contiene la parte di gestione dei tirocini (aggiunta, modifica, approvazione, dettaglio, candidatura)
	\item \textit{InternshipProposal}: contiene la parte di gestione dei tirocini (modifica, approvazione, dettaglio, tracciamento)
	\item \textit{Shared}: contiene i componenti comuni a questi sotto moduli (header, footer e sidebar)
\end{itemize}


\subsection{Servizi e recupero dei dati}
\label{client:services}
I moduli visti poco sopra contengono i componenti responsabili della visualizzazione dei dati, che tuttavia non si preoccupano di recuperare in autonomia. Essi si affidano infatti ad uno strato di servizi che interagiscono con il \gls{backend} dell'applicazione, i \textit{services}.
I \textit{services} di \angular~seguono lo standard del \gls{backend}, ovvero ognuno di essi si preoccupa di gestire una sola entità.
 \begin{figure}[!h] 
 	\centering    
 	\lstinputlisting{Chapter2/service.ts}
 	\caption[Esempio di \textit{service} \gls{frontend}]{Esempio di \textit{service} \gls{frontend}}
 	\label{fig:client-service}
 \end{figure}
All'interno dei componenti che contengono il \textit{template} verranno iniettati uno o più \textit{services} che permetteranno di recuperare i dati da visualizzare.

 \subsection{Supporto multi lingua}
 La gestione della localizzazione dell'app è gestita tramite il pacchetto \acrshort{npm} \textit{ngx-translate}, che fornisce dei componenti per la traduzione delle stringhe nei \textit{template} dei \textit{components} e nei relativi \textit{view-model}. Le traduzioni sono salvate in un file \acrshort{json}
 \begin{figure}[!h] 
	\centering    
	\lstinputlisting{Chapter2/globalization.json}
	\caption[Estratto del file di globalizzazione \gls{frontend}]{Estratto del file di globalizzazione \gls{frontend}}
	\label{fig:client-globalization}
\end{figure}
ed è possibile recuperare in differenti modi:
\begin{itemize}
	\item \textit{Pipe}: da utilizzare nel \textit{template} quando ci sono espressioni da valutare
	\item \textit{Directive}: da utilizzare nel \textit{template} quando la stringa da tradurre è cablata
	\item \textit{Service}: da utilizzare nel \textit{view-model} dei \textit{components}
\end{itemize}
 \begin{figure}[!h] 
	\centering    
	\lstinputlisting{Chapter2/globalization.ts}
	\caption[Localizzare un \textit{component} con \textit{ngx-translate}]{Localizzare un \textit{component} con \textit{ngx-translate}}
	\label{fig:client-ngxtranslate}
\end{figure}
Un servizio registrato al \textit{bootstrap} dell'applicazione si preoccupa di leggere la lingua corrente e caricare il file di traduzione relativo, oppure uno di \textit{fallback}.