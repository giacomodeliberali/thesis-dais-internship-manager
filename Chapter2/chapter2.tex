\chapter{Architettura}

Questo capitolo illustra il funzionamento e la struttura delle architetture software, sia per quanto riguarda il \gls{frontend} che il \gls{backend}. Dal momento che entrambe le applicazioni \angular~e \nodejs~operano sugli stessi oggetti, le loro definizioni risiedono in un pacchetto \gls{npm} condiviso che esporta tutte le entità necessarie, in modo da riutilizzare il codice e renderlo facilmente manutenibile. Le principali entità esportate sono:
\begin{itemize}
	\item \textit{User}: rappresenta un utente del sistema con uno specifico ruolo
	\item \textit{Company}: rappresenta un'azienda
	\item \textit{Internship}: rappresenta un'offerta di tirocinio di un'azienda
	\item \textit{InternshipProposal}: rappresenta una proposta di candidatura di uno studente per un tirocinio
\end{itemize}

\section{Architettura lato server}

Il \gls{backend} di \projectName è un'applicazione \nodejs~che si appoggia sul \gls{framework} \expressjs~per l'esposizione di \acrshort{rest} \acrshort{api} che interagiscono con \mongodb~attraverso l'\acrshort{orm} \mongoosejs. \\

\noindent
L'infrastruttura server è divisa in diversi livelli con responsabilità diverse che interagiscono fra loro che verranno discusse nei paragrafi seguenti.

\subsection{Schemas}
Pur utilizzando \mongodb~che è uno schema-less \acrshort{dbms} ho preferito appoggiarmi su un sistema che mi permettesse di validare i dati e la loro struttura. Per fare ciò ho utilizzato \mongoosejs, un \gls{orm} che mi permette di definire la struttura dei documenti nelle collezioni del database, mi fornisce metodi di validazione e manipolazione basati su oggetti.

Gli \textit{schemas} sono la definizione dei documenti delle collezioni del database come previsti da \mongoosejs. Essi contengono la definizione del documento, dei suoi campi e del loro tipo.
% File
\begin{figure}[!h] 
	\centering    
	\lstinputlisting{Chapter2/schema.ts}
	\caption[Esempio di \textit{schema} dell'applicazione \gls{backend}]{Esempio di \textit{schema} dell'applicazione \gls{backend}}
	\label{fig:server-schema}
\end{figure}
Possiamo notare che in figura \ref{fig:server-schema}, all'interno della definizione della struttura del documento, vi sono tre proprietà --- \textit{attendances, internship e status}. Ognuno dei campi ha tipo diverso:
\begin{itemize}
	\item \textit{attendances} è un array di oggetti con una properietà \textit{date} di tipo \textit{Date} obbligatoria
	\item \textit{internship} è una referenza di un altro schema (Internship), che verrà popolato automaticamente in fase di lettura (effettua un \gls{sqljoin} in automatico con il plugin \acrshort{npm} `mongoose-autopopulate`)
	\item \textit{status} è semplice numero
\end{itemize}
Nel caso l'applicazione cerchi di salvare un oggetto che non rispetti i vincoli imposti dallo \textit{schema} viene sollevata un'eccezione che impedisce di rendere inconsistente il database.

\subsection{Repositories}

I \textit{repositories} sono classi legate ad uno specifico oggetto che esportano operazioni su di esso. Interrogano uno o più \textit{schemas} per leggere, scrivere o aggregare dati, e contengono solamente la logica di accesso ai dati, senza nessuna logica di business (ad esempio il controllo dei permessi). Tutti i \textit{repositories} dervano dal \textit{BaseRepository} che esporta le operazioni di base --- \gls{crud} --- oltre che un metodo per eseguire query personalizzate.
% File
\begin{figure}[!h] 
	\centering    
	\lstinputlisting{Chapter2/repository.ts}
	\caption[Esempio di \textit{Repository}]{Esempio di \textit{repository} dell'applicazione \gls{backend}}
	\label{fig:server-repository}
\end{figure}
Ogni \textit{repository} può esporre ulteriori metodi personalizzati ed eventualmente accedere e ad altri \textit{repositories} iniettati dal sistema di \acrfull{di}. Il sistema di \acrshort{di} adottato dal sistema si basa sul pacchetto \acrshort{npm} `inversisy`, che fornisce anche un modo di aggirare le dipendenze circolari (ad esempio tra due \textit{repositories}) tramite un meccanismo di \textit{lazy-inject}.
\section{Archietettura lato client}
\subsection{Angular}
Lorem ipsum dolor sit amet, consectetur adipiscing elit. Sed vitae laoreet lectus. Donec lacus quam, malesuada ut erat vel, consectetur eleifend tellus. Aliquam non feugiat lacus. Interdum et malesuada fames ac ante ipsum primis in faucibus. Quisque a dolor sit amet dui malesuada malesuada id ac metus. Phasellus posuere egestas mauris, sed porta arcu vulputate ut. Donec arcu erat, ultrices et nisl ut, ultricies facilisis urna. Quisque iaculis, lorem non maximus pretium, dui eros auctor quam, sed sodales libero felis vel orci. Aliquam neque nunc, elementum id accumsan eu, varius eu enim. Aliquam blandit ante et ligula tempor pharetra. Donec molestie porttitor commodo. Integer rutrum turpis ac erat tristique cursus. Sed venenatis urna vel tempus venenatis. Nam eu rhoncus eros, et condimentum elit. Quisque risus turpis, aliquam eget euismod id, gravida in odio. Nunc elementum nibh risus, ut faucibus mauris molestie eu.
 Vivamus quis nunc nec nisl vulputate fringilla. Duis tempus libero ac justo laoreet tincidunt. Fusce sagittis gravida magna, pharetra venenatis mauris semper at. Nullam eleifend felis a elementum sagittis. In vel turpis eu metus euismod tempus eget sit amet tortor. Donec eu rhoncus libero, quis iaculis lectus. Aliquam erat volutpat. Proin id ullamcorper tortor. Fusce vestibulum a enim non volutpat. Nam ut interdum nulla. Proin lacinia felis malesuada arcu aliquet fringilla. Aliquam condimentum, tellus eget maximus porttitor, quam sem luctus massa, eu fermentum arcu diam ac massa. Praesent ut quam id leo molestie rhoncus. Praesent nec odio eget turpis bibendum eleifend non sit amet mi. Curabitur placerat finibus velit, eu ultricies risus imperdiet ut. Suspendisse lorem orci, luctus porta eros a, commodo maximus nisi.
 
 \subsection{Supporto multi lingua}
