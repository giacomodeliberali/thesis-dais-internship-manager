\chapter{Conclusioni}

L'applicazione, sebbene soddisfi i casi d'uso previsti, è da considerarsi come una versione non ancora in grado di ricoprire tutti gli scenari realmente esistenti nel processo di gestione dei tirocini. Quello che potrebbe completare \projectName~è l'integrazione dell'autenticazione con il sistema federato dell'ateneo e l'aggiunta del ruolo utente "Garante"  --- che potrebbe convergere con il ruolo di professore~--- che si occupa di accertare che il tirocinio richiesto da uno studente possa permettere il riconoscimento dei crediti. In questo momento, infatti, la candidatura ad un tirocinio è da considerarsi come extra-curricolare. 
%
Per permettere il riconoscimento di crediti formativi è necessario che l'azienda sia convenzionata con l'ateneo e che il contenuto del tirocinio sia valido e inerente al percorso di studi. Per questo si potrebbe pensare di creare una sezione dell'applicazione che permetta di convenzionare le aziende interessate, e infine di permettere al professore di verificare che il piano dell'azienda proposto per il tirocinio sia sufficiente al riconoscimento dei crediti richiesti dallo studente.

\section{Sviluppi futuri e nuove integrazioni}
I possibili sviluppi futuri dell'applicazione potrebbero dunque coinvolgere:
\begin{itemize}
	\item la gestione di stage anche curriculari, permettendo il riconoscimento dei crediti formativi universitari (CFU)
	\item l'autenticazione mediante il sistema federato dell'ateneo, permettendo un controllo degli utenti più restrittivo
	\item la possibilità di convenzionare le aziende online, favorendo il numero di imprese che interagiscono con l'ateneo
\end{itemize}
