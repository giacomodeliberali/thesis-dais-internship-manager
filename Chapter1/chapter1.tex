\chapter{Introduzione}

%********************************** %First Section  *************************************
\section{Da dove nasce questo progetto}

Il progetto \projectName nasce con l'intento di semplificare il processo di gestione degli stage universitari. Il sistema correntemente adottato dall'ateneo non permette un'efficace fruizione dei contenuti né da parte degli studenti né tanto meno dal punto di vista dei professori e delle aziende. L'intero sistema non è altro che una semplice interfaccia web che mostra agli studenti autenticati tutte le offerte pubblicate.

Il workflow da seguire per inserire, cercare e candidarsi ad un'offerta di tirocinio è pittosto macchinoso. Se un'azienda desidera proporre un'offerta di tirocinio prima di tutto deve essere convenzionata con l'ateneo, dopodiché deve inviare un'email alla segreteria che provvederà, una volta validato il contenuto dell'offerta, all pubblicazione dello stessa. Una volta pubblicata l'offerta sarà visibile dagli studenti, che potranno candidarsi contattando prima il professore e in seguito l'azienda, sempre mediante un contatto via email. 

Risulta quindi chiaro che sia necessaria una soluzione che permetta di automatizzare il più possibile questo processo, che tenga traccia dell'andamento del tirocinio e ne monitori lo stato.



%********************************** %Second Section  *************************************
\section{Scelte e vincoli tecnici} 

La soluzione deve essere fruibile da quanti più dispositivi possibili, e per raggiungere questo obbiettivo è stato scelto di sviluppare una \textit{Web Application}. Appoggiandosi infatti all'accessibilità offerta dal web sarà sufficiente mantenere un solo codebase per raggiungere tutti i dispositivi - computers, smartphones e tablets.

Un requsito di fondamentale importanza è quindi la responsività dell'applicazione, data la diversità dei dispositivi che si intende supportare. 
Inoltre, per favorire l'accessibilità dell'applicazione essa dovrà essere multi lingua e in questa prima versione dovrà supportare almeno l'\textit{italiano} e l'\textit{inglese}.

\subsection{Tecnologie adottate}
Dal momento che abbiamo deciso di puntare su un applicazione web, le tecnologie che andremo ad utilizzare per il frontend della soluzione saranno sicuramente \textit{web-based}, in particolare il framework \textit{Angular}, prodotto opensource di casa Google. 
Per quanto rigurada il backend dell'applicazione, che dovrà gestire database ed esporre servizi alla parte client, abbiamo optato per \textit{Node.js} --- ed in particolare il framework \textit{Express.js} per la creazione dei servizi --- e \textit{MongoDB}, un database documentale \acrshort{nosql} come \acrshort{dbms}.

\subsubsection{Node.js \& Express.js}
\nodejs è un ambiente open source e cross platform che esegue codice Javascript lato server.
\begin{quote}
	<<Node.js\textregistered ~ è un runtime Javascript costruito sul motore JavaScript V8 di Chrome. Node.js usa un modello I/O non bloccante e ad eventi, che lo rende un framework leggero ed efficiente. L'ecosistema dei pacchetti di Node.js, npm, è il più grande ecosistema di librerie open source al mondo.>> \cite{nodejs}
\end{quote}

Storicamente Javascript era utilizzato solamente per scripting client-side, spesso incluso all'interno delle pagine web dove veniva eseguito client-side nel browser dell'utente. \nodejs permette agli sviluppatori di utilizzare scriping server-side, eseguendo comandi che producono contenuto dinamico prima che venga inviato al browser client-side. \nodejs rappresenta il paradigma <<Javascript everywhere>>\cite{jseverywhere}, unificando lo sviluppo di applicazioni web attorno ad un unico linguaggio di programmazione piuttosto che separando  linguaggi per client e server-side.

\nodejs porta la programmazione orientata agli eventi ai server Web, consentendo lo sviluppo di server Web veloci in JavaScript. Gli sviluppatori possono creare server altamente scalabili senza utilizzare il threading, ma utilizzando un modello di programmazione event-drivern che utilizza i callback per segnalare il completamento di un'attività.


\subsubsection{MongoDB}
\subsubsection{Angular}