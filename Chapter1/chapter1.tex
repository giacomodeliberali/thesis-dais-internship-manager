\chapter{Introduzione}

%********************************** %First Section  *************************************
\section{Da dove nasce questo progetto}

Il progetto \projectName nasce con l'intento di semplificare il processo di gestione degli stage universitari. Il sistema correntemente adottato dall'ateneo non permette un'efficace fruizione dei contenuti né da parte degli studenti né tanto meno dal punto di vista dei professori e delle aziende. L'intero sistema è una semplice interfaccia web che mostra agli studenti autenticati tutte le offerte pubblicate.

Il workflow da seguire per inserire, cercare e candidarsi ad un'offerta di tirocinio è piuttosto macchinoso. Se un'azienda desidera proporre un'offerta di tirocinio prima di tutto deve essere convenzionata con l'ateneo, dopodiché deve inviare un'email alla segreteria che provvederà, una volta validato il contenuto dell'offerta, alla pubblicazione della stessa. Una volta pubblicata l'offerta sarà visibile dagli studenti che potranno candidarsi contattando prima il professore e in seguito l'azienda, sempre mediante un rapporto basato su email. 

Risulta quindi necessaria una soluzione che permetta di automatizzare il più possibile questo processo, che tenga traccia dell'andamento del tirocinio e ne monitori lo stato.



%********************************** %Second Section  *************************************
\section{Scelte e vincoli tecnici} 

La soluzione deve essere fruibile da quanti più dispositivi possibili e per raggiungere questo obbiettivo ho scelto di sviluppare un'applicazione web. Appoggiandosi infatti all'accessibilità offerta dal web sarà sufficiente mantenere un solo codebase per raggiungere tutti i dispositivi --- computers, smartphones e tablets.

Un requisito di fondamentale importanza è quindi la responsività dell'applicazione, data la diversità dei dispositivi che si intende supportare. 
Inoltre, per favorire l'accessibilità dell'applicazione, essa dovrà essere multi lingua e in questa prima versione dovrà supportare almeno l'\textit{italiano} e l'\textit{inglese}.

\section{Tecnologie adottate}
Dal momento che ho deciso di puntare su un applicazione web, le tecnologie che andrò ad utilizzare per il \textit{\gls{frontend}} della soluzione saranno sicuramente web-based, in particolare lo stack \textit{\gls{mean}}. Questo insieme applicativo è composto da
\begin{enumerate}
	\item Un \acrshort{dbms} basato su un database documentale \acrshort{nosql} (\mongodb)
	\item Un framework server-side per la creazione di applicazione web e rest \textit{\acrshort{api}} (\expressjs)
	\item Un framework client-side (\angular)
\end{enumerate}


\subsection{\mongodb}

TODO

\subsection{\nodejs}
\nodejs~è un ambiente open source e cross platform sviluppato a partire dal 2009 che permette di eseguire codice Javascript lato server.
\begin{quote}
	<<Node.js\textregistered ~ è un runtime Javascript costruito sul motore JavaScript V8 di Chrome. Node.js usa un modello I/O non bloccante e ad eventi, che lo rende un framework leggero ed efficiente. L'ecosistema dei pacchetti di Node.js, npm, è il più grande ecosistema di librerie open source al mondo.>> \cite{nodejs}
\end{quote}

\noindent
Storicamente Javascript era utilizzato solamente per scripting client-side, spesso incluso all'interno delle pagine web dove veniva eseguito client-side nel browser dell'utente. \nodejs~permette agli sviluppatori di utilizzare scriping server-side, eseguendo comandi che producono contenuto dinamico prima che venga inviato al browser client-side. \nodejs~rappresenta il paradigma <<Javascript everywhere>>\cite{jseverywhere}, unificando lo sviluppo di applicazioni web attorno ad un unico linguaggio di programmazione piuttosto che separando  linguaggi per client e server-side.

% Archietettura nodejs
La sua architettura è basata sul modello orientato agli eventi (\acrshort{eda}), ciò significa che \nodejs~richiede al sistema operativo su cui è in esecuzione di ricevere notifiche al verificarsi di determinati eventi, rimanendo in stato di \textit{sleep} fino al ricevimento di tale notifica. Questo pattern architetturale permette una forma di comunicazione non bloccante basata sull'\textit{asynchronous I/O} che per il programmatore finale si traduce nell'utilizzo di \textit{\gls{callback}}. Per segnalare la conclusione di un \textit{task I/O} infatti, \nodejs~invoca la corrispondende \textit{\gls{callback}}, una semplice funzione, alla quale vengono passati i risultati dell'operazione appena conclusa.
%Dettagli tecnici
\nodejs~opera in un processo single-thread utilizzando il pattern \textit{\acrshort{patternobserver}} per la sottoscrizione e la gestione degli eventi, ottenendo così performance adatte ad applicazioni altamente realtime. Processa le richieste in arrivo in un ciclo, chiamato \textit{\gls{eventloop}}, dove ogni connessione è una piccola allocazione di memoria heap anziché un nuovo processo o thread. Alla fine della registrazione della \textit{\gls{callback}} il server rientra in modo automatico nell'\textit{\gls{eventloop}}, a differenza di altri server orientati agli eventi e vi esce solo quando non vi sono ulteriori \textit{\gls{callback}} da eseguire.

% Vantaggi e svantaggi
I vantaggi che hanno portato \nodejs~ad avere una diffusione così ampia sono molti. Sicuramente possiamo notare che Javascript è un linguaggio ben conosciuto e largamente utilizzato, quindi la curva di apprendimento di questa tecnologia è molto più breve; offrendo inoltre la programmazione orientata agli eventi permette agli sviluppatori di creare server in grado di gestire un alto numero di richieste simultanee che siano facilmente scalabili senza l'utilizzo del \textit{threading}. Lo svantaggio principale di \nodejs~è la mancanza al supporto per la scalabilità verticale, determinata dalla sua architettura single-thread.

\subsection{\expressjs}
\expressjs~è un \textit{\gls{framework}} per costruire applicazioni web ed \textit{\acrshort{api}} basato sulla piattaforma \nodejs. Nel corso del tempo è devenuto lo standard de facto per i \textit{\gls{framework}} server di \nodejs. 
% Descrizione principale
Si presenta in modo minimale, offrendo un sottile livello applicativo che punti a velocizzare lo sviluppo senza tuttavia oscurare le funzionalità di \nodejs. \expressjs~è diviso in diversi moduli che possono essere innestati uno spora l'altro, rendendolo adatto ad ogni tipo di applicazione.  
% Principali funzionalità
Le sue funzionalità principali sono:
\begin{enumerate}[label=(\alph*)]
	\item Un sistema di routing: contenuto all'interno del pacchetto `express-router`, è il modulo per la gestione e la manipolazioni delle routes. Permette di definire in modo gerarchico un insieme di \textit{\acrshort{url}}, alle quali associare una specifica azione. Un'azione è un metodo che viene invocato e che produce una risposta.
	\item \textit{\acrshort{http}} helpers, come redirect o sistemi di caching: contenuto all'interno del modulo core, mette a disposizione alcune utility class che facilitano operazioni ripetitive oppure forniscono strumenti aggiuntivi utili ad ogni tipologia di sistema che si intende sviluppare
	\item Supporto a diversi template engines. Dal momento che si possono anche realizzare applicazioni che ritornano del contenuto, ad esempio un applicazione \textit{\gls{mvc}}, è necessario un interprete del template che permetta l'inserimento dinamico di contenuto all'interno di esso. Un esempio di quelli che \expressjs~supporta out of the box sono \textit{Pug (Jade), Haml.js, React, Blade} e altri.
\end{enumerate}

\noindent
Un esempio di un'applicazione \expressjs~che una volta avviata risponde 'Hello World!' è la seguente:
\lstinputlisting{Chapter1/express.js}
Qui possiamo vedere che una volta creata l'applicazione (riga 3), viene registrata una nuova route (riga 5) alla quale viene associata una \textit{\gls{callback}}. Questa \textit{\gls{callback}} riceve due parametri, la richiesta e la risposta. Il processo di \nodejs~resterà in stato di sleep fino a che una nuova richiesta verrà inoltrata nella route appena definita (quindi fino a che non verrà eseguita una chiamata in \textit{\acrshort{http} GET} all'indirizzo dove è in esecuzione l'applicazione). Una volta ricevuta la notifica \nodejs~entrerà nell'\textit{\gls{eventloop}} per gestirla e una volta completata l'operazione eseguirà la \textit{\gls{callback}} registrata, rispondendo al client che ha effettuato la connessione con la stringa 'Hello World!'.

\subsection{\angular}

TODO

\section{Attori del sistema}

Dal momento che vi sono più soggetti diversi che accedono alla piattaforma è necessario definire i ruoli dei soggetti coinvolti. Il sistema prevede la gestione di quattro tipi di account, ognuno con permessi diversi, e la personalizzazione dell'interfaccia in base all'utente correntemente autenticato.

\subsection{Azienda}
Inserisce le offerte di tirocinio presso una delle proprie sedi, visualizza le candidature ricevute e approva l'inizio di uno stage

\subsection{Professore}
Approva le offerte inserite dalle aziende prima che vengano pubblicate, approva e visualizza le richieste di candidatura degli studenti che li coinvolgono come referenti

\subsection{Studente}
Visualizza le offerte di tirocinio pubblicate e propone una candidatura indicando un professore

\subsection{Admin}
Ricopre il ruolo di ammistratore dei dati, è l'unico soggetto a non avere restrizioni sulla modifica o la cancellazione dei dati