\newdualentry{dbms} % label
{DBMS}            % abbreviation
{Database Management System}  % long form
{Un sistema software progettato per consentire la creazione, la manipolazione e l'interrogazione efficiente di database} % description

\newdualentry{nosql} % label
{NoSQL}            % abbreviation
{No Structured Query Language}  % long form
{Archivi di dati che il più delle volte non richiedono uno schema fisso (schemaless), evitano spesso le operazioni di giunzione (join) e puntano a scalare in modo orizzontale} % description

\newdualentry{eda}
{EDA}
{Event-driver architecture}
{Un pattern di architettura software che promuove la produzione, l'individuazione, il consumo e la reazione agli eventi}


\newdualentry{patternobserver}
{Observer}
{Observer pattern}
{Un design pattern in cui un oggetto, chiamato \textit{subject}, mantiene un elenco dei suoi dipendenti, chiamati \textit{observers} e li notifica automaticamente di qualsiasi cambiamento di stato, di solito chiamando uno dei loro metodi}


\newdualentry{api}
{API}
{Application Programming Interface}
{Un insieme di definizioni di metodi, protocolli e strumenti per la creazione di software applicativo. In termini generali, si tratta di un insieme di metodi di comunicazione definiti in modo chiaro tra i vari componenti software}

\newdualentry{url}
{URL}
{Uniform Resource Locator}
{Rappresenta una referenza ad una risorsa web. Specifica la sua posizione nella rete e il meccanismo per recuperarla}

\newdualentry{mvc}
{MVC}
{Model-View-Controller}
{
Un pattern architetturale comunemente usato per lo sviluppo di software che divide un'applicazione in tre parti interconnesse. Separa le rappresentazioni interne delle informazioni dal modo in cui vengono presentate all'utente. Il modello di progettazione MVC disaccoppia questi componenti principali consentendo un riutilizzo efficiente del codice
}

\newdualentry{http}
{HTTP}
{Hyper Text Transfer Protocol}
{Protocollo di comunicazione che sta alla base del World Wide Web, Supporta diversi metodi, detti verbi, quali \textit{GET, PUT, POST, PATCH, DELETE, OPTIONS, HEAD}}

\newdualentry{mean}
{MEAN}
{MongoDB, Express, Angular \& Node.js}
{Uno stack di tencologie utilizzate insieme al fine di creare applicazioni web}


\newdualentry{json}
{JSON}
{Javascript Object Notation}
{Una rappresentazione testuale che permette di codificare un oggetto \gls{javascript}}


\newdualentry{bson}
{BSON}
{Binary \acrshort{json}}
{Una rappresentazione binaria di \gls{json}}


\newdualentry{html}
{HTML}
{HyperText Markup Language}
{Un linguaggio di markup per la formattazione e impaginazione di documenti ipertestuali disponibili nel web}

\newdualentry{dom}
{DOM}
{Document Object Model}
{Una forma di rappresentazione dei documenti strutturati come modello orientato agli oggetti. Nativamente supportato dai browser per modificare gli elementi di un documento \acrshort{html}, DOM è un modo per accedere e aggiornare dinamicamente il contenuto, la struttura e lo stile dei documenti.}


\newdualentry{oop}
{OOP}
{Object-oriented programming}
{Un paradigma di programmazione che permette di definire oggetti software in grado si interagire gli uni con gli altri attraverso uno scambio di messaggi}


\newdualentry{di}
{DI}
{Dependency injection}
{Un design pattern della programmazione orientata agli oggetti il cui scopo è quello di semplificare lo sviluppo e migliorare la testabilità di software di grandi dimensioni}

% use with \gls{label}
\newglossaryentry
{framework}
{
	name={
		Framework
	},
	text={
		framework
	},
	description={
		Fornisce un modo standard per creare e distribuire applicazioni
	}
}

\newglossaryentry
{callback}
{
	name={
		Callback
	},
	text={
		callback
	},
	description={
		Rappresenta un codice eseguibile che viene passato come argomento ad un altra funzione da cui ci si aspetta che venga richiamata (eseguita) in un dato momento. L'esecuzione potrebbe essere immediata come nei callbacks sincroni oppure potrebbe verificarsi in un momento successivo, come nel caso dei callbacks asincroni
	}
}

\newglossaryentry
{eventloop}
{
	name={
		Event-loop
	},
	text={
		event-loop
	},
	description={
		Un costrutto di programmazione che attende e invia eventi o messaggi in un programma
	}
}

\newglossaryentry
{frontend}
{
	name={
		Front-end
	},
	text={
		front-end
	},
	description={
		Si intende la parte di applicazione visibile all'utente finale, che nasconde i dettagli implementetivi
	}
}

\newglossaryentry
{backend}
{
	name={
		Back-end
	},
	text={
		back-end
	},
	description={
		Si intende la parte di applicazione non visibile all'utente finale che manipola, gestisce e fornsce i dati alla parte di \gls{frontend}
	}
}

\newglossaryentry
{javascript}
{
	name={
		Javascript
	},
	text={
		javascript
	},
	description={
		Un linguaggio di scripting orientato agli oggetti e agli eventi, comunemente utilizzato nella programmazione Web lato client per la creazione, in siti web e applicazioni web, di effetti dinamici interattivi
	}
}

\newglossaryentry
{typescript}
{
	name={
		Typescript
	},
	text={
		typescript
	},
	description={
		Un linguaggio programmazione open-source sviluppato di Microsoft, super-set di \gls{javascript}, estende la sintassi di \gls{javascript} introducento lo static typing
	}
}

\newglossaryentry
{sqljoin}
{
name={
	JOIN
},
text={
	JOIN
},
description={
	Una clausola del linguaggio SQL che serve a combinare le tuple di due o più relazioni di un database tramite l'operazione di congiunzione dell'algebra relazionale
}
}